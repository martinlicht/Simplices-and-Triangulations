   \subsection{Translation, Linear Transformation and Affine Transformation}
      
      We define several classes of transformations that we frequently use.
      %sets\\
      linear, translation, affine transformation\\
      
      \begin{definition*}
        %$\textbf{Translation}$\\
         Let $\textbf{v}$ be a fixed vector, a $\textbf{translation}$ ${T}_v$ on a figure applies as ${T}_v (\textbf{p}) = \textbf{p} + \textbf{v}$, for a vector $\textbf{p}$ in the figure which we translate.
      \end{definition*}
      A $\textit{translation}$ moves every point of a figure or space by the same distance in the same direction. A translation ${T}$ can be represented by an addition of a constant vector to every point.\\


      \begin{definition*}
      %\paragraph{Linear transformation}
      Let ${V}$ and ${W}$ be vector spaces over the same field $\textbf{K}$. We say a function $\mathit{f}: {V} \rightarrow {W}$ is $\textbf{linear transformation}$ if the following is satisfied:\\
      \begin{align*}
      \mathit{f}(\textbf{u} + \textbf{v}) &= \mathit{f}(\textbf{u}) + \mathit{f}(\textbf{v}) \qquad \forall \textbf{u}, \textbf{v} \in{V},\\
      \mathit{f}(c\textbf{u}) &= c\mathit{f}(\textbf{u}), \qquad \forall \textbf{u} \in{V}, ~c\in\textbf{K}.
      \end{align*}
      \end{definition*}
      In other words, a linear transformation is a mapping between two vector spaces which preserves the operations of addition and scalar multiplication. Moreover, if ${V}$ and ${W}$ are finite dimensional, we can represent the linear transformation ${f}$ by a matrix ${M}$. For example, if ${M}$ is an ${m} \times {n}$ matrix, then ${f}$ is a linear transformation from $\mathbf{R}^n$ to $\mathbf{R}^m$. \\


      \begin{definition*}
      %\paragraph{Affine Transformation}
      An $\textbf{affine transformation}$ from $\mathbb{R}^n$ to $\mathbb{R}^n$ is of the form\\
      \begin{equation*}
      {F}(x) = {Ax} + {v}, \qquad {x}\in\mathbb{R}^n,
      \end{equation*}
      where ${A}\in\mathbb{R}^{n\times n}$ is a linear transformation, and  ${v}\in\mathbb{R}^n$ is a translation vector.\\
      \end{definition*}
      Affine transformation preserves points, lines and planes, but need not preserve point zero in a linear space in contrast to linear transformation. So we see that translation and linear transformation is affine, but the opposite is not true.\\
      %\indent
      The inverse mapping is only defined if $A^{-1}$ exists, and we define the inverse mapping $F^{-1} = {x} \mapsto {A}^{-1}({x} - {v})$ is also an affine transformation. Affine transformation helps carry results from one simplex to another simplex in our discussion, and more details are covered after introducing ${Simplices}$ and ${Triangulations}$ in the next section.


